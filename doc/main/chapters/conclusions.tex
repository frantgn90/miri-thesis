\chapter{Epilog}

On the future work and conclusions

\section{Future work}

Even if a considerable amount of hours have been invested in this project, as 
always, there are some parts to be improved to run from this first prototype to
a complete and well-polished analysis tool. In this section there are going to
explain some aspects that in our opinion should be improved or even extended.


In order to improve the 
overall performance of an application we have to focus on improving the part of
the application that where more time is being wasting on. Trace reduction presents
to be the phase that is highly dominating the overall execution time explaining 
about the 90\% of the time. Improve this phase is then a big deal. For sure
there should be some improvements to do in the current version, that reads
sequentially the whole trace but the real business is on parallelize this
process. The sequential read of the trace is not critical for our purposes since
every event have a timestamp attached to the order can be reconstructed whenever
we want. One possible proposal is to split trace into several files and then
parse every one of the independently in parallel, once done the partial results 
could be merged. Additionally a possible solution could be to work with some
big data frameworks like Hadoop what can provide all the reduction
infrastructure.

The exposed methodology relies on the mpi calls clustering for extract the
applications' structure, so it is a key piece of the overall proposal. The fact
of select the best features set to perform this clustering is a difficult
business. Our first proposal have been to use number of dynamic iterations and
iterations mean time but as has been explained, in some applications this
features leads to aliasing so additional checks have been developed in order to
detect them. Having this drawback in mind we have driven an analysis of different
features that could improve the behaviour but without positives. More work have 
to be done in this sense trying to reduce the cases where aliasing appears by
means of analyze in a deeper way new features to use.

About the scaling of clustering phase, also should be explored the alternative
to merge same mpi calls from different ranks at the reduce step. It will drive
to lighter clusterings since the number of items will be reduced dramatically
for high processes count executions. In principle it
have been discarded because we taken a conservative point of view so 
just in case same mpi call in different processes behave different but
since our approach is very focused on SPMD applications can be assumed they will
behave similar.

Finally something to explore is to improve the output by developing a graphical
user interface that allow to interact with the results in a more natural way.

\section{Conclusions}

In this work have been proposed and developed a new approach for dealing with
the extraction of the applications' structure. After revise the literature in
this field it has been decided to attack this problem from a different point of
view what is indeed the main contribution of this thesis. Here the structure 
detection problem has been considered as a classification problem instead of 
a sequential pattern mining one. The main reason to do that is because we are 
exclusively
focused on HPC applications so we decided to exploit their 
idiosyncrasy that have leads to a sort of ad-hoc methodology for HPC applications 
that allows to decrease the complexity and so improve the scalability.  

The correctness of its results have been demonstrated for the studied cases.
Additionally scalability study have demonstrated how the applications scales in
terms of unique mpi calls. The conclusion is that when the execution is scaled
in terms of problem size, the number of unique mpi calls remains the same. The
reason is because the application is just executing more iterations of the most
outer loop. When the application is scaling in terms of processes the number of
unique mpi calls grows at least with the same factor as the number of processes.
These results allows us to say the method is not only correct but also scalable.

To conclude, even if some aspects of the proposal needs to be depurated and more
research has to be done in others, in general it have been satisfactory.

\section{Acknowledgments}


