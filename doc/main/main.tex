\documentclass[12pt]{report}

\usepackage[utf8]{inputenc}
\usepackage[english]{babel}
\usepackage{graphicx}

%Options: Sonny, Lenny, Glenn, Conny, Rejne, Bjarne, Bjornstrup
%\usepackage[Sonny]{fncychap}
\usepackage{amsmath}
\usepackage[colorlinks=true, allcolors=blue]{hyperref}
\usepackage{hyperref}
\usepackage[a4paper,top=3cm,bottom=3cm,left=3cm,right=3cm]{geometry}

\graphicspath{ {imgs/} }

\title{
  {\Huge Inferring program structure from an execution trace}
}
\author{
  {Juan Francisco Martínez Vera}\\
  {\tt juan.martinez@bsc.es}
}
\date{\today}

\begin{document}
\maketitle
\tableofcontents

\chapter{Introduction}
A typical configuration for high performance machines are clusters. It means a t
eam of individual processors or multiprocessors (and lately more specific hardwa
re like GPUs) working together, interconnected by a super-fast network. The fund
amental idea behind these big machines is getting speedup by mean of partitionin
g the problem and parallelize the execution. So all processors (or a subset) in 
the cluster will be dealing with different parts of the same problem and communi
cating between them in order to end up with a solution. For example imagine we h
ave a weather forecast software, in order to speedup the forecast we can partiti
on the surface of the earth and hold every portion to one individual processor. 
The communication will be needed because the weather will also depend on surroun
dings, so every partition will need also information of other partitions results.

In Computer Science the discipline in charge to drive research in this field is 
the High Performance Computing, i.e. HPC. HPC has been increasing in importance 
and nowadays can be said it is the third support of science with theory and math
ematics. The science that can be done thanks to this big machines goes from eart
hquake predictions to the analysis of the DNA of a carcinogenic cell, from weath
er forecast to material physics simulation. HPC research is not just limited to 
the hardware layer but drive research in all layers in the Transformation Hierar
chy\cite{transformationHierarchy} in figure \ref{transformationHierarchyImg}.

\begin{figure}
  \caption{Levels of transformation}
  \label{transformationHierarchyImg}
  \centering
    \includegraphics[width=120px]{transformationhierarchy.png}
\end{figure}

\begin{figure}
  \caption{Gordon Moore's prediction done in 1965}
  \label{moore-prediction}
  \centering
    \includegraphics[width=200px]{moore_prediction.png}
\end{figure}

Resources are limited and expensive, so we have to be smart and use them in an
efficient way. Improving the efficiency is not just a matter that affects to one
layer of the stack but can be applied to every one of them. For example the
tremendous evolution of the last fourty years were improvements done mostly on
circuits layer what have been following the Moore's law \cite{moore:1965}. The
manufacturers have been reducing the size of transistors by a ration of 2 every
18 months as were predicted in figure \ref{moore-prediction}. Performance improvements 
were also at microarchitecture level with
disruptive designs that allows ILP like HPS\footnote{High Performance Substrace,
what is indeed out-of-order execution with in-order retirement.}
\cite{Patt:1985:HNM:18927.18916}, speculative execution with prefetchers, 
branch prediction or even memory access value speculation,
VLIW\footnote{Very Long Instruction Word} or TLP\footnote{Thread Level
Parallelism} with multi-threading. Improvements on memory hierarchy like cache 
associativitym, non-blocking cache or trace cache. The last
big revolution affected both architecture and program layers and was the 
multicore revolution. Was specially important because until then, programmers were 
agnostics, just feel their programs went faster but now, programs have to be 
architecture aware. With this last revolution a new effort for hide machine stuff
from programmers arise and end up with programming models like OpenMP
\cite{openmpwww}, OpenACC\cite{openaccwww} or OmpSs\cite{ompsswww}. A relatively
recent example is shown is this paper \cite{Alvarez:2015:CPT:2872887.2750411}
where they are trying to automatically use the fast scraptchpad memories by
means of compilers procedures and extra hardware support instead
of let programmers to deal with this piece of hardware directly. The need to be 
always faster and faster is even more pressing in HPC that is in fact an
important science driver.

This thesis is centered on program layer\footnote{The program layer 
bring together the program itself, programming models, frameworks and libraries.}. 
This layer is the actual implementation of the algorithm, i.e. where programmers
transforms algorithm to  semantic code that can be transformed lately into
executable binary by compilers. This layer is indeed the interface between pure 
algorithm and the machine architecture. Having an efficient algorithm
mathecamically speaking is not enforcing having an efficient program 
because at algorithm level we do not care about the machine. 
The mechanism to assess the quality in terms of performance are the {\bf performance 
analysis tools} that allows to analyze and detect the bottlenecks. 
There are two main trends, by one hand we have the profilers and by other hand trace
tools. Profiler tools provide performance summaries. This summaries provides 
coarse-grain information, with low-overhead. It means that you can feel the problem 
and infer what could be the origin but the information is not enough if you want 
to be sure. For gather more detail, the other typical approach are the tracing 
tools. High accurate and detailed information can be obtained, but it needs 
more resources in terms of cpu, disk and analyst effort because more data implies 
more complexity. This demand of resources becomes worst since the capacity in 
terms of parallelism is increasing, so the number of tasks to monitor makes 
tracing by one hand hardly scalable and by the other hand tricky to analyze. 

Researchers are currently facing this two drawbacks. There are currently several
specialists driving research in the scalability field, they are trying to reduce 
the overhead of tracing in terms of time and trace sizes by mean of several 
techniques like machine learning, data-mining and so on like in 
\cite{llort2015intelligent}. About analysis field, the complexity of the analysis 
can be overcome by adding intelligence to performance analysis tools, last 
trend is to provide an automatic smart layer that automate an important part 
of the analysis and help the analyzers in their work. Some examples of this 
research line are automatic performance analysis \cite{wolf2003automatic}, 
automatic structure extraction \cite{casas2007automatic}, phases detection 
\cite{gonzalez2013application} fundamental factors models \cite{casas2008aass}, 
automatic analysis throw deep learning \cite{simon:2017:perfdp} and so on.

This thesis is focused on the analysis field and is devoted to humbly contribute 
to ease the work of the analyzers by
automate part of the analysis. The goal of this work is to automatically detect
the internal structure of an application and correlate it with performance metrics by
mean of a post-mortem trace analysis that will help analyzers to have a general
overview about what is actually going on. So instead of deal with whole
information from the very beggining, this approach let the analyzer to figure
out to what part of the gathered information focus on, speeding up the process of
analysis. Also helps to correlate performance analysis with source code, 
since the detected application structure would be similar as
code structure and will contain source coude information that will allow to
easly pinpoint one metric to the code. The structure representation will ignore 
implementation details what therefore will contribute to faster and therefore better 
quality reports for developers. The implementation has been done also thinking 
in techniques of trace size reductions, in particular sampling, so is prepared 
for future developments where the traces do not contain the whole execution behavior.

\section{Motivation}

High performance computers becomes more complex every generation. The number of 
processors is increasing dramatically, e.g. the current number one on the Top500
 list\footnote{Top500 is the list of the top 500 most powerful high performance 
computers in the world. It is updated two times per year.} is the Sunway TaihuLight 
with 10,649,600 cores\cite{top5002017}. Analyze application with this quantity 
of processes with fine-grain detail becomes a really tough task when not
impossible because of the huge amount of information. The typical tools like 
visualizers are not enough. The researchers demands tools that ease the process 
of performance analysis therefore the main motivation of this work is to ease 
this analysis by means of reducing execution traces to the minimal and meaningful 
expression.

% Maybe in other section. Talk about SPMD and Burst Syncrhonized

We can say that HPC applications shares a common idiosyncrasy between them. 
Mathematic solvers needs to iterate again and again until the result converge and 
simulation software are used to be programed to evolve over timesteps. So in 
general, HPC applications consists on a big outer loop\footnote{This same expression 
has been used in other related works} that is being executed again and again the
same code but with evolving data. It means that the whole huge trace could be 
reduced to a minimum pseudo-code expression that represents just one iteration 
with aggregated data for the whole loop so its somehow folding all the iterations
space into just one. This pseudo-code express indeed the actual internal 
structure of the application.

There are previous works (discussed widely on section \ref{related_work})  that 
tries to represent the internal structure of an application but they are used to
 use directed graphs or just callstack trees. In this thesis the proposal is to 
generate a pseudo-code with loop and conditional structures that can represent w
ith more or less accuracy the source code. The motivation to use pseudo-code is 
to get insights about the internal structure of the application that can ease th
e task of correlate performance metrics to source code.

\section{Previous and related work}\label{related_work}

Blah Blah Blah

\section{Performance analysis tools environment}

Blah Blah Blah

% NOTES
% .- Remember to demonstrate the interarrival time of different calls in the sam
%e loop body tend to be the same.


\bibliography{bibliography}
\bibliographystyle{apalike}
\end{document}
